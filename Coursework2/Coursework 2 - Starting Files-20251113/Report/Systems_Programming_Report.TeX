\documentclass{article}
\usepackage{graphicx}
\usepackage{blindtext}
\usepackage{amsmath}

\title{Systems Programming Report 2}
\author{James Henrik Middleton}
\date{November 2025}

\begin{document}
\maketitle

\begin{flushleft}
    I have implemented Tier three of the Solution.\\
    This code compiles correctly with no bugs or errors that I am aware of.\\
    Note that you must compile using the syntax \\"g++ -std=c++17 -pthread strace-analyser.cpp -o strace-analyser"\\
    to correctly compile using threads on the school linux servers.\\
\end{flushleft}

\begin{figure}[htbp!]
    \begin{center}
    The timings for the Sequential version:\\
    \includegraphics[width=0.2\linewidth]{./timings-sequential.png}
    \end{center}
\end{figure}
\hfill
\begin{center}
    The timings for the Threaded version:
    \begin{tabular}{||c c c c c||} 
        \hline
        Runs/Threads & 1 & 2 & 4 & 8 \\ [0.5ex] 
        \hline\hline
        1 & 0.231s & 0.319s & 0.432s & 0.525s\\ 
        \hline
        2 & 0.213 & 0.330s & 0.482s & 0.496s\\
        \hline
        3 & 0.220 & 0.319s & 0.468s & 0.516s\\
        \hline
        Median & 0.221 & 0.323 & 0.461 & 0.512\\
        \hline
    \end{tabular}
\end{center}

\begin{flushleft}
    Despite the threaded method theoretically being more efficient it has a significantly longer 
    runtime than the sequential version. This is most likely due to the sample test file being so 
    small. The threads don't have time to work properly and the time that it takes to open and close
    them makes it inefficient for small files.\\
\end{flushleft}

\end{document}